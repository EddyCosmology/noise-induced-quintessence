\documentclass[preprint]{aastex7}
\makeatletter
\renewcommand{\frontmatter@title@above}{}
\makeatother
\usepackage{graphicx}
\usepackage{amsmath,amssymb}
\usepackage{natbib}
\usepackage{booktabs} % For tables

\shorttitle{Noise-Induced Transitions in Quintessence DE}
\shortauthors{Thornton}

\begin{document}

\title{Noise-Induced Transitions in Quintessence-Like Dark Energy: Sensitivity to Stochastic Vacuum Fluctuations}

\author[0009-0001-9080-8200]{Micah David Thornton}
\affiliation{Independent Researcher}
\email{eddycosmology@gmail.com}
\date{Draft version February 20, 2026}
\footnote{This preprint is licensed under a Creative Commons Attribution 4.0 International License (CC BY 4.0). To view a copy of this license, visit http://creativecommons.org/licenses/by/4.0/ or send a letter to Creative Commons, PO Box 1866, Mountain View, CA 94042, USA.}

\begin{abstract}
We present a numerical exploration of how multiplicative stochastic noise affects the late-time behavior of a minimally coupled scalar field dark energy model motivated by DESI DR2 hints for dynamical dark energy (\( w_0 > -1 \), \( w_a < 0 \)). Building on the phenomenological framework of Thornton (2026), which includes nonlinear advection, higher-derivative hyperdiffusion, and a running vacuum term yielding \( w(z=0) \approx -0.86 \) in a tuned low-noise regime, we systematically vary the noise amplitude \( \sigma \). Through ensembles of 50,000 realizations per \( \sigma \), we uncover a sharp transition: at low \( \sigma \) (\( \lesssim 0.02 \)), the field remains frozen near \( w \approx -1 \); at moderate \( \sigma \) (\( \sim 0.05 \)), \( w(0) \) shifts to \( \sim -0.85 \); at higher \( \sigma \) (\( \gtrsim 0.1 \)), the mean becomes positive (\( w > 0 \)), destroying acceleration. Parameter tuning (\( \beta \to 1 \), \( \kappa \to 0.1 \), adjusted initial \( \phi \)) extends the viable window, keeping \( w(0) \lesssim -0.85 \) up to \( \sigma \approx 0.05 \) with low ensemble scatter. We estimate a critical noise threshold \( \sigma_c \approx 0.06 \) (tuned) where \( w(0) \) crosses \( -1/3 \). The results constrain vacuum fluctuation strength at cosmological scales, with implications for stochastic gravity, objective collapse models, and tests with Euclid, LSST, and CMB-S4 through stochastic non-Gaussianity (\( f_{\rm NL} \sim 10 \)--50).
\end{abstract}

\keywords{evolving dark energy --- cosmology: theory --- stochastic processes --- large-scale structure of universe --- quintessence --- stochastic noise --- objective collapse models --- vacuum fluctuations --- DESI DR2 --- non-Gaussianity}

\section{Introduction}
Recent DESI Data Release 2 (DR2, 2025) analyses show growing evidence for evolving dark energy, with $w_0 w_a$CDM fits favoring \( w_0 > -1 \) and \( w_a < 0 \) at 2.8--4.2\( \sigma \) significance depending on supernova samples \citep{DESI2025a,DESI2025b,DESI2025c}. This evolution helps alleviate aspects of the Hubble tension and motivates exploration of dynamical dark energy mechanisms beyond a pure cosmological constant.

Stochastic effects in scalar fields have been studied in inflation and early-universe contexts \citep{Starobinsky1994,Grain2010}, but their role in late-time dark energy remains underexplored. In a companion work (Thornton 2026), we proposed a phenomenological scalar field model incorporating nonlinear advection, higher-derivative hyperdiffusion, multiplicative stochastic noise, and a running vacuum term to address vacuum suppression and match DESI hints. Here, we systematically analyze the sensitivity of the late-time equation of state \( w(z=0) \) to the noise amplitude \( \sigma \). Using large numerical ensembles, we demonstrate a noise-induced phase-transition-like behavior, from a frozen \( \Lambda \)-like attractor at low noise to a positive-\( w \) regime at high noise. We identify parameter adjustments that enhance robustness and discuss implications for constraints on vacuum fluctuations and future observational tests.

\section{Model and Numerical Setup}

The effective equation of motion for the scalar field \( \phi \) is (in conformal time, rescaled units):

\begin{equation}
\ddot{\phi} + 3\mathcal{H}\dot{\phi} + V'(\phi) + \beta \phi \dot{\phi}^4 + \kappa (\Box \phi)^2 = \eta(t) \frac{\phi^2}{\phi_0^2},
\end{equation}

with \( V(\phi) = \frac{1}{2} m^2 \phi^2 \) and running vacuum \( \Lambda(H) = \Lambda_0 + 3\nu H^2 \). The noise \( \eta(t) \) is Gaussian white noise with variance \( \sigma / \sqrt{\Delta t} \).

We solve using the Euler-Maruyama scheme with \( \Delta t = 0.005 \) from \( a = 10^{-3} \) to \( a=1 \). Ensembles of 50,000 independent realizations are computed for each \( \sigma \), ensuring Monte Carlo error \( < 0.01 \) on means (verified via jackknifing). The code is implemented in Python with NumPy and is publicly available at [https://github.com/eddycosmology/noise-induced-quintessence].

We compare two parameter sets (Table \ref{tab:params}):

\begin{table}[htbp]
\centering
\begin{tabular}{lcc}
\toprule
Parameter & Original & Tweaked \\
\midrule
\( \beta \) & 10 & 1.0 \\
\( \kappa \) & 0.01 & 0.1 \\
Initial \( \phi \) (z \( \gg \) 1) & 5.0 & 4.0 \\
\bottomrule
\end{tabular}
\caption{Parameter sets. Tweaked values reduce nonlinearity and enhance damping.}
\label{tab:params}
\end{table}

\section{Results}

\subsection{Deterministic Limit}

In the absence of noise, both sets yield \( w(0) \approx -0.9999 \), mimicking a cosmological constant.

\subsection{Noise Sensitivity}

Figure \ref{fig:sensitivity} shows ensemble-averaged \( w(0) \) vs. \( \sigma \):

\begin{figure*}
\centering
\includegraphics[width=0.9\textwidth]{w_DE_comparison_tweaked_vs_original.png}
\caption{Ensemble-averaged effective dark energy equation-of-state parameter \( w_{\rm DE}(z=0) \) as a function of the multiplicative noise amplitude \( \sigma \) (Gaussian white noise variance \( \sigma / \sqrt{\Delta t} \) in the rescaled Euler-Maruyama integration). Results are from 50,000 independent realizations per \( \sigma \) value (Monte Carlo error on means \( <0.01 \), verified via jackknifing). 
Blue circles with error bars: tuned parameters (\( \beta=1.0 \), \( \kappa=0.1 \), initial \( \phi(z\gg1)=4.0 \)), which enhance damping and extend the viable acceleration regime; orange squares: original baseline parameters (\( \beta=10 \), \( \kappa=0.01 \), initial \( \phi=5.0 \)). Green plus symbol: deterministic (zero-noise) limit, yielding \( w(0) \approx -0.9999 \) (frozen, \( \Lambda \)-like behavior). 
Grey dashed horizontal line: constant \( w = -1 \) (cosmological constant). Red dotted horizontal line: \( w = -1/3 \) (boundary for no late-time acceleration, \( \Omega_{\rm DE} \) ceases to drive expansion). 
The tuned parameter set maintains \( w(0) \lesssim -0.85 \) (consistent with DESI DR2 combined-probe hints favoring \( w_0 > -1 \) and \( w_a < 0 \) at 2.8--4.2\( \sigma \) significance, depending on supernova datasets; e.g., $\sim$3.1\( \sigma \) with CMB alone) up to \( \sigma \approx 0.05 \)--0.06, with low ensemble scatter (\( \sigma_w \sim 0.02 \)--0.11). Beyond a critical threshold \( \sigma_c \approx 0.06 \) (tuned; estimated where mean \( w(0) \) crosses \( -1/3 \)), stochastic kicks overpower potential damping and hyperdiffusion, driving the field away from the freezing attractor toward positive \( w \) (no acceleration). This constrains cosmological-scale vacuum fluctuation amplitudes in stochastic gravity and objective collapse models (e.g., implying CSL-like rates \( \lambda_{\rm CSL} \lesssim 10^{-20} \) Hz). The low-noise regime (\( \sigma \lesssim 0.02 \)) aligns with the fiducial results of Thornton (2026), where \( w(z) \) evolves smoothly from near \( -1 \) at high redshift to \( \approx -0.86 \) locally with extremely low scatter (\( \lesssim 0.001 \)). Error bars show $1\sigma$ ensemble scatter; implications include enhanced stochastic non-Gaussianity (\( f_{\rm NL} \sim 10 \)--50) testable with CMB-S4 lensing and mild cluster abundance shifts (\( \Delta N(>10^{14} M_\odot)/N \sim 5\% \)) in LSST.}
\label{fig:sensitivity}
\end{figure*}

At low \( \sigma = 0.02 \), the tweaked set gives \( w(0) = -0.976 \pm 0.020 \), robustly \( \Lambda \)-like. At \( \sigma = 0.05 \), tuned \( w(0) = -0.849 \pm 0.111 \), still accelerating and consistent with DESI dynamical hints. At \( \sigma = 0.10 \), both transition to positive \( w \). We estimate a critical \( \sigma_c \approx 0.06 \) (tuned) where mean \( w(0) = -1/3 \).

Figure \ref{fig:hist} shows \( w(0) \) distributions at \( \sigma = 0.05 \), illustrating reduced scatter in the tuned case.


\begin{figure}
\centering
\includegraphics[width=\columnwidth]{w_distribution_sigma_0.050.png}
\caption{Marginalized posterior distribution of the effective dark energy equation-of-state parameter \( w_{\mathrm{DE}}(z=0) \) (or instantaneous \( w(0) \)) derived from a Monte Carlo simulation of 50,000 samples, employing tuned cosmological parameters with a characteristic uncertainty scale \( \sigma = 0.050 \). The blue histogram represents the probability density from the tuned model, peaking near \( w \approx -0.85 \) to \( -0.9 \) (indicative of dynamical dark energy or quintessence-like behavior). Vertical reference lines are shown as follows: dashed at \( w = -1 \) (cosmological constant, $\Lambda$CDM) and dotted at \( w = -1/3 \) (limit for no late-time acceleration, corresponding to matter-dominated deceleration). The extended tail toward less negative---and slightly positive---values, including the small residual bump at \( w \gtrsim 0 \), results from degeneracies among parameters in dynamical dark energy models (e.g., the CPL parametrization \( w(z) = w_0 + w_a z/(1+z) \) or equivalent forms), where MCMC or posterior sampling explores the low-probability wing permitted by the data, priors, and likelihood surface.}
\label{fig:hist}
\end{figure}

\subsection{Analytic Insight}

The transition can be understood as noise overpowering the potential minimum: for \( \sigma \ll \kappa \), damping dominates; for \( \sigma > \sigma_c \sim \sqrt{\kappa m^2 / \beta} \), stochastic kicks destabilize freezing.

\section{Discussion}
The noise-induced transition constrains vacuum fluctuation amplitudes: \( \sigma \lesssim 0.05 \) to preserve acceleration in the tuned regime, implying CSL rates \( \lambda_{\rm CSL} \lesssim 10^{-20} \) Hz at cosmic scales. This links quantum collapse models to cosmological observables and complements the low-noise fiducial results of Thornton (2026), where \( w(z) \) evolves consistently with DESI DR2 hints.

Implications include enhanced \( f_{\rm NL} \sim 10 \)--50 in CMB lensing (CMB-S4 testable at $3\sigma$) and cluster abundance deviations \( \Delta N(> M)/N \sim 5\% \) for \( M > 10^{14} M_\odot \) in LSST. Limitations: Euler-Maruyama scheme used showing sufficient and convergent results; higher order Milstein tested but unstable due to implicit solve interaction. Future work: adaptive stepping, full perturbation analysis. Broader impact: Demonstrates how stochastic vacuum noise can mimic dynamical DE signatures -- relevant for interpreting DESI DR2 and designing future surveys.


\section*{Acknowledgments}

This work benefited from assistance provided by Grok, a large language model developed by xAI, which was used for python code generation, equation and numerical simulation refinement, literature reference suggestions and formatting, and help drafting sections of this manuscript. All scientific interpretations, model design, parameter choices, and final conclusions are the sole responsibility of the human author.

\bibliographystyle{aasjournal}
\bibliography{refs}
\nocite{Thornton2026main}
\end{document}